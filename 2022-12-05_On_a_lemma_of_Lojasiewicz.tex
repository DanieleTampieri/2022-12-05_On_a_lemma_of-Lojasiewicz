 %********************************************************************
%           On a lemma of Lojasiewicz
%********************************************************************
\documentclass[a4paper,10pt]{article}
%********************************************************************
%            Packages
%********************************************************************
\usepackage[english]{babel}
\usepackage[utf8]{inputenc}
\usepackage[intlimits]{amsmath}
\usepackage{amsfonts}
\usepackage{amssymb}
\usepackage{amsthm}
\usepackage{IEEEtrantools}
\usepackage{mathrsfs}
\usepackage{mathtools}
\usepackage{stix}
\usepackage{cite}
\usepackage{romannum}
\newcommand{\dm}{\operatorname{d\!}}
\newcommand{\Bsy}[1]{\ensuremath{{\boldsymbol{#1}}}}
\newcommand{\St}{\mathbb{St}}
\newtheorem{cor}{Corollary}
\newtheorem{defn}{Definition}
\newtheorem{lemma}{Lemma}
\newtheorem{obs}{Observation}
\newtheorem{theo}{Theorem}
\usepackage{hyperref}
%********************************************************************
%            Content
%********************************************************************
\title{On a lemma of Łojasiewicz}

\author{Daniele Tampieri}
\date{}

\begin{document}
\maketitle
\pagenumbering{arabic}
\begin{abstract}
  The purpose of this short note is to make better known (and prove a slightly extended version of) a lemma on the boudary behavior of analytic functions. This lemma seems to have been proved for the first time by Stanisław Łojasiewicz in his paper (possibly the very first) \cite{Lojasiewicz1950} on the standard Fatou-Riesz theorem.
\end{abstract}
\section{Basic definitions}
\begin{defn} Let $A,B\subset \Bbb C$ two subsets in the complex plane: their \emph{(Euclidean) distance} $\rho:\mathscr{P}(\Bbb C)\times\mathscr{P}(\Bbb C)\to\Bbb R_{\ge0}$ is defined as
  \begin{equation*}\label{eq:dist}
    \rho(A,B)=\inf_{z_1\in A \wedge z_2\in B}|z_1-z_2|
  \end{equation*}
\end{defn}
In the sequel we will deal only with the distance between a point $z$ inside a domain $G$ and its boundary $\partial G$ or the distance between two points $z_1, z_2\in G$: thus, by abuse of notation, we respectively write $\rho(\{z\},\partial G)\triangleq \rho_{\partial G}(z)$ and $\rho(\{z_1\},\{z_2\})\triangleq \rho(z_1,z_2)$.
\begin{defn}\label{def:appreg} Let $G\in\Bbb C$ be a domain, $\zeta_0 \in\partial G$ a point on its boundary, $M>1$, $0< \varepsilon \le 1$ and $\theta\in[0,2\pi]$ three real numbers. A $\theta$-\emph{directed $(\varepsilon,M)$-conical approach region with vertex $\zeta_0$} is a set $C_{\theta}^{\varepsilon}(\zeta_0,M)\subset\Bbb C$ defined as
  \begin{equation}
    \begin{split}
      C_{\theta}^{\varepsilon}(\zeta_0,M)=\bigg\{ z\in \Bbb C  : &\; \big(|z -\zeta_0|<\varepsilon\big) \\
      & \land\Big(-\arccos\tfrac{1}{M}\le\arg (\zeta_0- z) -\theta \le + \arccos\tfrac{1}{M}\Big)\bigg\}
    \end{split}
  \end{equation}
  such that its closure is included in the closure of $G$, i.e. $\overline{C_{\theta}^{\varepsilon}(\zeta_0,M)}\subset\overline{G}$.
\end{defn}
The following definition is inspired by the analysis of the Abel--Stolz lemma in Knopp {\rm\cite[§54, pp.~406--407]{Knopp1951}} (see also the original work by Stolz and Gmeiner \cite[§\Romannum{4}.15, pp.~287--288]{StolzGmeiner1905}).
\begin{defn}\label{def:locst} Let $G\in\Bbb C$ be a domain, $\zeta_0 \in\partial G$ a point on its boundary, $M>1$, $0< \varepsilon \le 1$ and $\theta\in[0,2\pi]$ three real numbers. A $\theta$-\emph{directed $(\varepsilon,M)$-Stolz region} with vertex $\zeta_0$, is a closed subset of the closure of $G$ defined as
  \begin{equation}\label{eq:appreg}
    \begin{split}
      \St_\theta^\varepsilon\big(\zeta_0,M\big) = \bigg\{ z\in \Bbb C  : &\; {\left(\big| 2(z - \zeta_0) + {\varepsilon}{ e^{i\theta}}\big|\le \varepsilon \right)} \\
      & \land \left[{|\zeta_0-z|}\le M{\left({\varepsilon}/{2} - \left|z - \zeta_0 + {\varepsilon}\tfrac{ e^{i\theta}}{2}\right|\right)}\right]\bigg\},
    \end{split}
  \end{equation}
  where $M>1$.
\end{defn}
\begin{obs} Below we prove two properties involving $\theta$-directed $(\varepsilon,M)$-conical approach and Stolz regions with the same vertex $\zeta_0$ that will be useful in the sequel.
  \item $\overline{ \St_\theta^\varepsilon\big(\zeta_0,M\big)}\subsetneq\overline{ C_{\theta}^{\varepsilon}(\zeta_0,M)}$ and both regions are compactly contained in $\overline{G}$.  To see this, first note that $z\in\St_\theta^\varepsilon\big(\zeta_0,M\big)$ implies 
    \[
      \big| 2(z - \zeta_0) + {\varepsilon}{ e^{i\theta}}\big|\le \varepsilon \iff \big| (z - \zeta_0) + \frac{\varepsilon}{2}{ e^{i\theta}}\big|\le \frac{\varepsilon}{2}
    \]
    and thus
    \[
      \frac{\varepsilon}{2} \ge \big| (z - \zeta_0) + \frac{\varepsilon}{2}{ e^{i\theta}}\big|\ge | z - \zeta_0 | - \frac{\varepsilon}{2} \iff |z -\zeta_0|\le \varepsilon
    \]
    Moreover $z\in\St_\theta^\varepsilon\big(\zeta_0,M\big)$ also implies that
    \[
      \begin{split}
        {|\zeta_0-z|} & \le M{\left({\varepsilon}/{2} - \left|z - \zeta_0 + {\varepsilon}\tfrac{ e^{i\theta}}{2}\right|\right)} \\
        & \iff \frac{|\zeta_0-z|}{{\varepsilon}/{2} - \left|z - \zeta_0 + {\varepsilon}\tfrac{ e^{i\theta}}{2}\right|}\le M\\
        & \iff \frac{\frac{2}{\varepsilon}|\zeta_0-z|}{1 - \left|1 - \frac{2}{\varepsilon}(\zeta_0-z)e^{-i\theta}\right|}\le M
      \end{split}
    \]
    and by putting $\arg(\zeta- z)-\theta\triangleq\varphi$ and $\frac{2}{\varepsilon}|\zeta_0-z|\triangleq \rho$ we get
    \[
      \begin{split}
        {|\zeta_0-z|}&\le M{\left({\varepsilon}/{2} - \left|z - \zeta_0 + {\varepsilon}\tfrac{ e^{i\theta}}{2}\right|\right)} \\
        & \iff \frac{\rho}{1 - \sqrt{1-2\rho\cos\varphi +\rho^2}}\le M\\
        & \iff \frac{\rho}{M}\le {1 - \sqrt{1-2\rho\cos\varphi +\rho^2}} \\
        & \iff \left(1 - \frac{\rho}{M} \right) \le 1-2\rho\cos\varphi +\rho^2\\
        & \iff \cos\varphi \ge \frac{1}{M} + \frac{\rho}{2} \left(1 -\frac{1}{M^2}\right) \ge \frac{1}{M}
      \end{split}
    \]
  \item The limiting arguments for $\zeta_0 - z$ as: in a less formal manner, the two regions are tangent at the vertex $\zeta_0$.  
\end{obs}

The following definitions are inspired by the one given in {\rm\cite[§1.1, p.~8]{DiBiase1998}}.
\begin{defn}\label{def:ntlim} Let $f: G\to \Bbb C$ be a holomorphic function and $\zeta_0\in\partial G$ a point on the boundary of its domain of definition: $f$ is said to have a \emph{non-tangential limit} as $z\to\zeta_0$ if and only if there exists a conical approach region $C_{\theta}^{\varepsilon}(\zeta_0,M)$ such that $\lim_{t\to 1}f(\gamma(t))=s$ along all continuous curves $\gamma: [0,1]\to\Bbb C$ such that
  \begin{equation*}
    \begin{cases}
      \gamma(1)=\zeta_0, \\
      \gamma([0,1])\subseteq C_{\theta}^{\varepsilon}(\zeta_0,M).
    \end{cases}
  \end{equation*}
\end{defn}

\section{The fundamental lemma}
The following lemma was proved by Stanisław Łojasiewicz in what is perhaps his first published paper.
\begin{lemma}{\rm\cite[lemme II, p.~242]{Lojasiewicz1950}}\label{lemma:Loj} Let $G\in\Bbb C$ be a domain, $\zeta_0 \in\partial G$ a point on its boundary, $g(z)$ a holomorphic function on $G$ such that its limit for $z \to\zeta_0$ exists and finally let $\rho(z)$ be the distance between $z$ and $\partial G$: then
$$
\left(z-\zeta_0\right) \cdot g^{\prime}(z) \underset{z \to \zeta_0}{\longrightarrow} 0
$$
in such a way that the quotient $\left|\zeta_0-z\right| / \rho(z)$ remains bounded.
\end{lemma}
\proof Let  $s=\lim _{z \to\zeta_0} g(z)$, $C_z=\big\{\zeta \in \mathbb{C}: | \zeta-z \mid=\frac{\rho(z)}{2}\big\}$ and $\eta(z)=\max _{{C}_z}|g(z)-s|$. For all $z \in G$,
$$
g^{\prime}(z)=\frac{1}{2 \pi i} \int_{C_z} \frac{g(\zeta)}{(\zeta-z)^2} \operatorname{d}\!\zeta=\frac{1}{2 \pi i} \int_{C_z}\frac{g(\zeta)-s}{(\zeta-z)^2} \operatorname{d}\! \zeta \text {, }
$$
thus it follows that
$$
\left|g^{\prime}(z)\right| \leqslant \frac{1}{2 \pi} \pi \rho(z) \frac{\eta(z)}{\left(\frac{1}{2} \rho(z)\right)^2}=2 \frac{\eta(z)}{\rho(z)}
$$
and consequently
$$
\left|\left(z-\zeta_0\right) \cdot g^{\prime}(z)\right| \leqslant 2 \frac{\left|z-\zeta_0\right|}{\rho(z)} \eta(z)
$$
which implies the thesis since $\eta(z) \to 0$ for $z \to \zeta_0$. \qed

The following lemma is a generalization of the previous one, in that it requires only the existence a of non tangential limit as $z$ tends toward the boundary point $\zeta_0$ of the function $g(z)$ 

\begin{lemma}\label{lemma:main} Let $G\in\Bbb C$ be a domain, $\zeta_0 \in\partial G$ a point on its boundary, $g(z)$ a holomorphic function on $G$ such that its non tangential limit for $z \to\zeta_0$ exists and finally let $\rho(z)$ be the distance between $z$ and $\partial G$: then

$$
\left(z-\zeta_0\right) \cdot g^{\prime}(z) \underset{z \to \zeta_0}{\longrightarrow} 0
$$
in such a way that the quotient $\left|\zeta_0-z\right| / \rho(z)$ remains bounded.
\end{lemma}
\proof Let  $s=\lim _{z \to\zeta_0} g(z)$, $C_z=\big\{\zeta \in \mathbb{C} :| \zeta-z \mid=\frac{\rho(z)}{2}\big\}$ and $\eta(z)=\max _{{C}_z}|g(z)-s|$. For all $z \in G$,
$$
g^{\prime}(z)=\frac{1}{2 \pi i} \int_{C_z} \frac{g(\zeta)}{(\zeta-z)^2} \operatorname{d}\!\zeta=\frac{1}{2 \pi i} \int_{C_z}\frac{g(\zeta)-s}{(\zeta-z)^2} \operatorname{d}\! \zeta \text {, }
$$
thus it follows that
$$
\left|g^{\prime}(z)\right| \leqslant \frac{1}{2 \pi} \pi \rho(z) \frac{\eta(z)}{\left(\frac{1}{2} \rho(z)\right)^2}=2 \frac{\eta(z)}{\rho(z)}
$$
and consequently
$$
\left|\left(z-\zeta_0\right) \cdot g^{\prime}(z)\right| \leqslant 2 \frac{\left|z-\zeta_0\right|}{\rho(z)} \eta(z)
$$
which implies the thesis since $\eta(z) \to 0$ for $z \to \zeta_0$. \qed

\section{Final notes and observations}
In this section I list some notes about the two lemmas proved above,
\begin{itemize}
\item Despite the fact that non tangential limit is a weaker concept than limit, the second lemma above cannot be said to be a generalization, since the the boundary point $\zeta_0$ can be choosen as the vertex of a cusp. Then lemma~\ref{lemma:main} is not applicable as in this case there is not an appoach region with vertex in $\zeta_0$ fully containded in the interior of $G$, while lemma~\ref{lemma:Loj} is perfectly applicable.
\item This lemma clarifies a doubt that I expressed in a MathOverflow question~\cite{396814}. The assumption on the growth of the function that seemed too specific and somewhat artificial to me, now appears quite natural in the light of Łojasiewicz's lemma~\ref{lemma:Loj}. Giovanni Ricci~(see the references cited in~\cite{396814}) seems simply chosing an explicit instance of a general behavior in order to show how to approach the general case by using his method.
\end{itemize}

\bibliographystyle{plain}
\bibliography{2022-12-05_Lojasiewicz}
\end{document}