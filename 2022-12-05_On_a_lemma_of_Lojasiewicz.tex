%********************************************************************
%           On a lemma of Lojasiewicz
%********************************************************************
\documentclass[a4paper,10pt]{article}
%********************************************************************
%            Packages
%********************************************************************
\usepackage[english]{babel}
\usepackage[utf8]{inputenc}
\usepackage[intlimits]{amsmath}
\usepackage{amsfonts}
\usepackage{amssymb}
\usepackage{amsthm}
\usepackage{IEEEtrantools}
\usepackage{mathrsfs}
\usepackage{stix}
\usepackage{cite}
\newcommand{\dm}{\mathrm{d}}
\newcommand{\Bsy}[1]{\ensuremath{{\boldsymbol{#1}}}}
\newcommand{\St}{\mathbb{St}}
\newtheorem{cor}{Corollary}
\newtheorem{defn}{Definition}
\newtheorem{lemma}{Lemma}
%\newtheorem{stackrel}
\newtheorem{theo}{Theorem}
\usepackage{hyperref}
%********************************************************************
%            Content
%********************************************************************

\title{On a lemma of Łojasiewicz}

\author{Daniele Tampieri}
\date{}

\begin{document}
\maketitle
\begin{abstract}
  The purpose of this short note is to make better known and slightly extend a lemma used by Stanisław Łojasiewicz in his proof of the standard Fatou-Riesz theorem.
\end{abstract}
\section{Basic definitions}
Let's start by giving some definitions and notations.
\begin{defn} Let $A,B\subset \Bbb C$ two subsets in the complex plane: their \emph{(Euclidean) distance} $\rho:\mathscr{P}(\Bbb C)\times\mathscr{P}(\Bbb C)\to\Bbb R_{>0}$ is defined as
  \begin{equation*}\label{eq:dist}
    \rho(A,B)=\inf_{z_1\in A \wedge z_2\in B}|z_1-z_2|
  \end{equation*}
\end{defn}
In the sequel we will consider exclusively the distance between a point $z$ in the interior of a domain $G$ and the boundary $\partial G$ of that domain: therefore we simplify the notation by putting $\rho(z,\partial G)\triangleq \rho(z)$.
\begin{defn} Let $G\in\Bbb C$ be a domain, $\zeta_0 \in\partial G$ a point on its boundary. A non tangential approach region along the direction
  \begin{equation}\label{eq:appreg}
    \Gamma\St
  \end{equation}
\end{defn}
The preceeding definition is inspired by the one given in {\rm\cite[§1.1, p.~8]{DiBiase1998}}.

\section{The fundamental lemma}

\begin{lemma}\label{lemma:main} Let $G\in\Bbb C$ be a domain, $\zeta_0 \in\partial G$ a point on its boundary, $g(z)$ a holomorphic function on $G$ such that its non tangential limit for $z \to\zeta_0$ exists and finally let $\rho(z)$ be the distance between $z$ and $\partial G$: then

$$
\left(z-\zeta_0\right) \cdot g^{\prime}(z) \underset{z \to \zeta_0}{\longrightarrow} 0
$$
in such a way that the quotient $\left|\zeta_0-z\right| / \rho(z)$ remains bounded.
\end{lemma}
\proof Let  $s=\lim _{z \to\zeta_0} g(z)$, $C_z=\big\{\zeta \in \mathbb{C}|| \zeta-z \mid=\frac{\rho(z)}{2}\big\}$ and $\eta(z)=\max _{{C}_z}|g(z)-s|$. For all $z \in G$,
$$
g^{\prime}(z)=\frac{1}{2 \pi i} \int_{C_z} \frac{g(\zeta)}{(\zeta-z)^2} \operatorname{d}\!\zeta=\frac{1}{2 \pi i} \int_{C_z}\frac{g(\zeta)-s}{(\zeta-z)^2} \operatorname{d}\! \zeta \text {, }
$$
thus it follows that
$$
\left|g^{\prime}(z)\right| \leqslant \frac{1}{2 \pi} \pi \rho(z) \frac{\eta(z)}{\left(\frac{1}{2} \rho(z)\right)^2}=2 \frac{\eta(z)}{\rho(z)}
$$
and consequently
$$
\left|\left(z-\zeta_0\right) \cdot g^{\prime}(z)\right| \leqslant 2 \frac{\left|z-\zeta_0\right|}{\rho(z)} \eta(z)
$$
which implies the thesis since $\eta(z) \to 0$ per $z \to \zeta_0$. \qed


\begin{cor}{\rm\cite[lemme II, p.~242]{Lojasiewicz1950}} Assumming the same hypotheses of lemma~\ref{lemma:main}
\end{cor}

\section{Final notes and observations}

\bibliographystyle{plain}
\bibliography{2022-12-05_Lojasiewicz}
\end{document}