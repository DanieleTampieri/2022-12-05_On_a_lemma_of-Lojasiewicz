%********************************************************************
%           On a lemma of Lojasiewicz
%********************************************************************
\documentclass[a4paper,10pt]{article}
%********************************************************************
%            Packages
%********************************************************************
\usepackage[english]{babel}
\usepackage[utf8]{inputenc}
\usepackage[intlimits]{amsmath}
\usepackage{amsfonts}
\usepackage{amssymb}
\usepackage{amsthm}
\usepackage{IEEEtrantools}
\usepackage{mathrsfs}
\usepackage{stix}
\usepackage{cite}
\newcommand{\dm}{\mathrm{d}}
\newcommand{\Bsy}[1]{\ensuremath{{\boldsymbol{#1}}}}
\newcommand{\St}{\mathbb{St}}
\newtheorem{cor}{Corollary}
\newtheorem{defn}{Definition}
\newtheorem{lemma}{Lemma}
%\newtheorem{stackrel}
\newtheorem{theo}{Theorem}
\usepackage{hyperref}
%********************************************************************
%            Content
%********************************************************************

\title{On a lemma of Łojasiewicz}

\author{Daniele Tampieri}
\date{}

\begin{document}
\maketitle
\begin{abstract}
  The purpose of this short note is to make better known and slightly extend a lemma used by Stanisław Łojasiewicz in his proof of the standard Fatou-Riesz theorem.
\end{abstract}
\section{Basic definitions}
Let's start by giving some definitions and notations.
\begin{defn} Let $A,B\subset \Bbb C$ two subsets in the complex plane: their \emph{(Euclidean) distance} $\rho:\mathscr{P}(\Bbb C)\times\mathscr{P}(\Bbb C)\to\Bbb R_{>0}$ is defined as
  \begin{equation*}\label{eq:dist}
    \rho(A,B)=\inf_{z_1\in A \wedge z_2\in B}|z_1-z_2|
  \end{equation*}
\end{defn}
In the sequel we will consider exclusively the distance between a point $z$ in the interior of a domain $G$ and the boundary $\partial G$ of that domain: therefore we simplify the notation by putting $\rho(z,\partial G)\triangleq \rho(z)$.
\begin{defn} Let $G\in\Bbb C$ be a domain, $\zeta_0 \in\partial G$ a point on its boundary. A $\varepsilon$-\emph{Stolz region} with vertex $\zeta_0$ directed along the direction $\theta\in[0,2\pi]$, is a closed subset of the closure of $G$ such that with the following structure
\begin{equation}\label{eq:appreg}
  \St_\theta^\varepsilon\big(\zeta_0,M\big) = \left\{ z\in\Bbb D(z_0,R): {|\zeta_b-z|}\le M{\big| 1-|z-\zeta_0+e^{i\theta}|\big|}\right\},
\end{equation}
where $M>1$ e $z_b\in\partial\Bbb D(z_o,R)$ è un punto della circonferenza. Anche in questo caso, la notazione
verrà progressivamente semplificata se $z_0=0$ poi $R=1$ e infine se $z_b=1$:
\begin{IEEEeqnarray}{rCl}
  \St\big(z_b,M,\Bbb D(0,R) \big) & = & \left\{ z\in\Bbb D(R): {|\zeta_0-z|}\le M{| R-|z||}\right\} \triangleq \St\big(z_b, M, \Bbb D(R) \big) \nonumber\\
  \St\big(z_b,M, \Bbb D(1)\big) & = & \left\{ z\in\Bbb D: {|z_b-z|}\le M{| R -|z||}\right\} \triangleq \St(z_b,M)\nonumber\\
    \St(1,M) & = & \left\{ z\in\Bbb D: {|1-z|}\le M{|1 - |z||}\right\} \triangleq \St(M).\nonumber
\end{IEEEeqnarray}
\end{defn}
The preceeding definition is inspired by the one given in {\rm\cite[§1.1, p.~8]{DiBiase1998}}, as it is the following one.
\begin{defn}\label{def:ntlim}
\end{defn}
\section{The fundamental lemma}
The following lemma was proved by Stanisław Łojasiewicz in what is perhaps his first published paper.

\begin{lemma}{\rm\cite[lemme II, p.~242]{Lojasiewicz1950}}\label{lemma:Loj} Let $G\in\Bbb C$ be a domain, $\zeta_0 \in\partial G$ a point on its boundary, $g(z)$ a holomorphic function on $G$ such that its limit for $z \to\zeta_0$ exists and finally let $\rho(z)$ be the distance between $z$ and $\partial G$: then
$$
\left(z-\zeta_0\right) \cdot g^{\prime}(z) \underset{z \to \zeta_0}{\longrightarrow} 0
$$
in such a way that the quotient $\left|\zeta_0-z\right| / \rho(z)$ remains bounded.
\end{lemma}
\proof Let  $s=\lim _{z \to\zeta_0} g(z)$, $C_z=\big\{\zeta \in \mathbb{C}: | \zeta-z \mid=\frac{\rho(z)}{2}\big\}$ and $\eta(z)=\max _{{C}_z}|g(z)-s|$. For all $z \in G$,
$$
g^{\prime}(z)=\frac{1}{2 \pi i} \int_{C_z} \frac{g(\zeta)}{(\zeta-z)^2} \operatorname{d}\!\zeta=\frac{1}{2 \pi i} \int_{C_z}\frac{g(\zeta)-s}{(\zeta-z)^2} \operatorname{d}\! \zeta \text {, }
$$
thus it follows that
$$
\left|g^{\prime}(z)\right| \leqslant \frac{1}{2 \pi} \pi \rho(z) \frac{\eta(z)}{\left(\frac{1}{2} \rho(z)\right)^2}=2 \frac{\eta(z)}{\rho(z)}
$$
and consequently
$$
\left|\left(z-\zeta_0\right) \cdot g^{\prime}(z)\right| \leqslant 2 \frac{\left|z-\zeta_0\right|}{\rho(z)} \eta(z)
$$
which implies the thesis since $\eta(z) \to 0$ for $z \to \zeta_0$. \qed

The following lemma requires only the exitence a of non tangential limit of the involved  function $g(z)$ as $z$ moves toward the boundary point $\zeta_0$

\begin{lemma}\label{lemma:main} Let $G\in\Bbb C$ be a domain, $\zeta_0 \in\partial G$ a point on its boundary, $g(z)$ a holomorphic function on $G$ such that its non tangential limit for $z \to\zeta_0$ exists and finally let $\rho(z)$ be the distance between $z$ and $\partial G$: then

$$
\left(z-\zeta_0\right) \cdot g^{\prime}(z) \underset{z \to \zeta_0}{\longrightarrow} 0
$$
in such a way that the quotient $\left|\zeta_0-z\right| / \rho(z)$ remains bounded.
\end{lemma}
\proof Let  $s=\lim _{z \to\zeta_0} g(z)$, $C_z=\big\{\zeta \in \mathbb{C} :| \zeta-z \mid=\frac{\rho(z)}{2}\big\}$ and $\eta(z)=\max _{{C}_z}|g(z)-s|$. For all $z \in G$,
$$
g^{\prime}(z)=\frac{1}{2 \pi i} \int_{C_z} \frac{g(\zeta)}{(\zeta-z)^2} \operatorname{d}\!\zeta=\frac{1}{2 \pi i} \int_{C_z}\frac{g(\zeta)-s}{(\zeta-z)^2} \operatorname{d}\! \zeta \text {, }
$$
thus it follows that
$$
\left|g^{\prime}(z)\right| \leqslant \frac{1}{2 \pi} \pi \rho(z) \frac{\eta(z)}{\left(\frac{1}{2} \rho(z)\right)^2}=2 \frac{\eta(z)}{\rho(z)}
$$
and consequently
$$
\left|\left(z-\zeta_0\right) \cdot g^{\prime}(z)\right| \leqslant 2 \frac{\left|z-\zeta_0\right|}{\rho(z)} \eta(z)
$$
which implies the thesis since $\eta(z) \to 0$ for $z \to \zeta_0$. \qed

\section{Final notes and observations}
In this section I list some notes about the two lemmas proved above,
\begin{itemize}
\item Despite the fact that non tangential limit is a weaker concept than limit, the second lemma above cannot be said to be a generalization, since the the boundary point $\zeta_0$ can be choosen as the vertex of a cusp. Then lemma~\ref{lemma:main} is not applicable as in this case there is not an appoach region with vertex in $\zeta_0$ fully containded in the interior of $G$, while lemma~\ref{lemma:Loj} is perfectly applicable.
\item This lemma clarifies a doubt that I expressed in a MathOverflow question~\cite{396814}. The assumption on the growth of the function that seemed too specific and somewhat artificial to me, now appears quite natural in the light of Łojasiewicz's lemma~\ref{lemma:Loj}. Giovanni Ricci~(see the references cited in~\cite{396814}) seems simply chosing an explicit instance of a general behavior in order to show how to approach the general case by using his method.
\end{itemize}

\bibliographystyle{plain}
\bibliography{2022-12-05_Lojasiewicz}
\end{document}